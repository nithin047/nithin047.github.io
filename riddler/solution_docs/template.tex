\documentclass[11pt]{article}       
\usepackage{amsmath,amsfonts,amssymb, amsthm}
\usepackage[hidelinks]{hyperref}
\usepackage{geometry}
\geometry{a4paper, portrait, margin=1in}
\newtheorem{theorem}{Theorem}[section]
\newtheorem{corollary}{Corollary}[theorem]
\newtheorem{lemma}[theorem]{Lemma}
\usepackage{graphicx}
\usepackage{float}
\usepackage{xcolor}
\usepackage[normalem]{ulem}
\theoremstyle{definition}
\newtheorem{definition}{Definition}[section]

\title{Riddler Solutions}
\date{August 19, 2022}
\author{Nithin Ramesan}


\begin{document}
	\maketitle
	\section{Riddler Express}
	Let $\mathcal{I}$\% be the specific interest rate that guarantees that the Rule of 72 is exactly correct. Let $t$ be the actual doubling time for this interest rate, and let $t'$ be the doubling time predicted by the Rule of 72. We can then write:
	\begin{align}
		\left(1 + \frac{\mathcal{I}}{100}\right)^t = 2,
	\end{align}
\begin{align}
	t' = \frac{72}{\mathcal{I}},
\end{align}
and
\begin{align}
	t=t'.
\end{align}
Combining these equations and with some rearrangement of terms, we get
\begin{align}
	\label{eq:final}
	\frac{\ln 2}{\ln \left(1 + \frac{\mathcal{I}}{100}\right)} = \frac{72}{\mathcal{I}},
\end{align}
where $\ln$ is the natural logarithm. (\ref{eq:final}) can be solved numerically via any number of root-finding algorithms to finally yield the required value of interest rate:
\begin{center}
	$\mathcal{I}=7.847$, or the interest rate is $7.847$\%.
\end{center}
\subsection{Where does the Rule of 72 come from?}
As a side exercise, I was curious how one arrives at the Rule of 72. It's easy to see by looking at the right-hand side of (\ref{eq:final}) and using the approximation $\ln(1+x)\approx x$ for small $x$:
\begin{align*}
	\frac{\ln 2}{\ln \left(1 + \frac{\mathcal{I}}{100}\right)} &\approx \frac{100\ln 2}{\mathcal{I}}\\
	&= \frac{69.31}{\mathcal{I}}.
\end{align*}
Picking $72$ as the numerator instead of $69.31$ makes sense because $72$ is divisible by most whole numbers less than $10$ (which is the range interest rates tend to fall in): $1, 2, 3, 4, 6, 8$ and $9$. $5 \text{ and } 7$ can be handled by assuming that $69.31 \approx 70$.
\section{Riddler Classic}
\end{document}